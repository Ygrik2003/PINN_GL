\documentclass[a4paper,14pt,oneside,openany]{memoir}

\tolerance10000
\hbadness10000

%%% Задаем поля, отступы и межстрочный интервал %%%

\usepackage[left=30mm, right=10mm, top=20mm, bottom=20mm]{geometry} % Пакет geometry с аргументами для определения полей
\pagestyle{plain} % Убираем стандарные для данного класса верхние колонтитулы с заголовком текущей главы, оставляем только номер страницы снизу по центру
\parindent=1cm % Абзацный отступ 1.25 см, приблизительно равно пяти знакам, как по ГОСТ
\usepackage{indentfirst} % Добавляем отступ к первому абзацу
%\linespread{1.3} % Межстрочный интервал (наиболее близко к вордовскому полуторному) - тут вместо этого используется команда OnehalfSpacing*

%%% Задаем языковые параметры и шрифт %%%

\usepackage[english, russian]{babel}                % Настройки для русского языка как основного в тексте
\babelfont{rm}{Times New Roman}                     % TMR в качестве базового roman-щрифта

%%% Задаем стиль заголовков и подзаголовков в тексте %%%

\setsecnumdepth{subsection} % Номера разделов считать до третьего уровня включительно, т.е. нумеруются только главы, секции, подсекции
\renewcommand*{\chapterheadstart}{} % Переопределяем команду, задающую отступ над заголовком, чтобы отступа не было
\renewcommand*{\printchaptername}{} % Переопределяем команду, печатающую слово "Глава", чтобы оно не печалось
%\renewcommand*{\printchapternum}{} % То же самое для номера главы - тут не надо, номер главы оставляем
\renewcommand*{\chapnumfont}{\normalfont\bfseries} % Меняем стиль шрифта для номера главы: нормальный размер, полужирный
\renewcommand*{\afterchapternum}{\hspace{1em}} % Меняем разделитель между номером главы и названием
\renewcommand*{\printchaptertitle}{\normalfont\bfseries\centering\MakeUppercase} % Меняем стиль написания для заголовка главы: нормальный размер, полужирный, центрированный, заглавными буквами
\setbeforesecskip{20pt} % Задаем отступ перед заголовком секции
\setaftersecskip{20pt} % Ставим такой же отступ после заголовка секции
\setsecheadstyle{\raggedright\normalfont\bfseries} % Меняем стиль написания для заголовка секции: выравнивание по правому краю без переносов, нормальный размер, полужирный
\setbeforesubsecskip{20pt} % Задаем отступ перед заголовком подсекции
\setaftersubsecskip{20pt} % Ставим такой же отступ после заголовка подсекции
\setsubsecheadstyle{\raggedright\normalfont\bfseries}  % Меняем стиль написания для заголовка подсекции: выравнивание по правому краю без переносов, нормальный размер, полужирный

%%% Задаем параметры оглавления %%%

\addto\captionsrussian{\renewcommand\contentsname{Содержание}} % Меняем слово "Оглавление" на "Содержание"
\setrmarg{2.55em plus1fil} % Запрещаем переносы слов в оглавлении
%\setlength{\cftbeforechapterskip}{0pt} % Эта команда убирает интервал между заголовками глав - тут не надо, так красивее смотрится
\renewcommand{\aftertoctitle}{\afterchaptertitle \vspace{-\cftbeforechapterskip}} % Делаем отступ между словом "Содержание" и первой строкой таким же, как у заголовков глав
\renewcommand*{\chapternumberline}[1]{} % Делаем так, чтобы номер главы не печатался - тут не надо
\renewcommand*{\cftchapternumwidth}{1.5em} % Ставим подходящий по размеру разделитель между номером главы и самим заголовком
\renewcommand*{\cftchapterfont}{\normalfont\MakeUppercase} % Названия глав обычным шрифтом заглавными буквами
\renewcommand*{\cftchapterpagefont}{\normalfont} % Номера страниц обычным шрифтом
\renewcommand*{\cftchapterdotsep}{\cftdotsep} % Делаем точки до номера страницы после названий глав
\renewcommand*{\cftdotsep}{1} % Задаем расстояние между точками
\renewcommand*{\cftchapterleader}{\cftdotfill{\cftchapterdotsep}} % Делаем точки стандартной формы (по умолчанию они "жирные")
\maxtocdepth{subsection} % В оглавление попадают только разделы первыхтрех уровней: главы, секции и подсекции

%%% Выравнивание и переносы %%%

%% http://tex.stackexchange.com/questions/241343/what-is-the-meaning-of-fussy-sloppy-emergencystretch-tolerance-hbadness
%% http://www.latex-community.org/forum/viewtopic.php?p=70342#p70342
\tolerance 1414
\hbadness 1414
\emergencystretch 1.5em                             % В случае проблем регулировать в первую очередь
\hfuzz 0.3pt
\vfuzz \hfuzz
%\dbottom
%\sloppy                                            % Избавляемся от переполнений
\clubpenalty=10000                                  % Запрещаем разрыв страницы после первой строки абзаца
\widowpenalty=10000                                 % Запрещаем разрыв страницы после последней строки абзаца
\brokenpenalty=4991                                 % Ограничение на разрыв страницы, если строка заканчивается переносом

%%% Объясняем компилятору, какие буквы русского алфавита можно использовать в перечислениях (подрисунках и нумерованных списках) %%%
%%% По ГОСТ нельзя использовать буквы ё, з, й, о, ч, ь, ы, ъ %%%
%%% Здесь также переопределены заглавные буквы, хотя в принципе они в документе не используются %%%

\makeatletter
    \def\russian@Alph#1{\ifcase#1\or
       А\or Б\or В\or Г\or Д\or Е\or Ж\or
       И\or К\or Л\or М\or Н\or
       П\or Р\or С\or Т\or У\or Ф\or Х\or
       Ц\or Ш\or Щ\or Э\or Ю\or Я\else\xpg@ill@value{#1}{russian@Alph}\fi}
    \def\russian@alph#1{\ifcase#1\or
       а\or б\or в\or г\or д\or е\or ж\or
       и\or к\or л\or м\or н\or
       п\or р\or с\or т\or у\or ф\or х\or
       ц\or ш\or щ\or э\or ю\or я\else\xpg@ill@value{#1}{russian@alph}\fi}
\makeatother

%%% Задаем параметры оформления рисунков и таблиц %%%

\usepackage{graphicx, caption, subcaption} % Подгружаем пакеты для работы с графикой и настройки подписей
\graphicspath{{src/}} % Определяем папку с рисунками
\captionsetup[figure]{font=small, width=\textwidth, name=Рисунок, justification=centering} % Задаем параметры подписей к рисункам: маленький шрифт (в данном случае 12pt), ширина равна ширине текста, полнотекстовая надпись "Рисунок", выравнивание по центру
% \captionsetup[subfigure]{font=small} % Индексы подрисунков а), б) и так далее тоже шрифтом 12pt (по умолчанию делает еще меньше)
\captionsetup[table]{singlelinecheck=false,font=small,width=\textwidth,justification=justified} % Задаем параметры подписей к таблицам: запрещаем переносы, маленький шрифт (в данном случае 12pt), ширина равна ширине текста, выравнивание по ширине
\captiondelim{ --- } % Разделителем между номером рисунка/таблицы и текстом в подписи является длинное тире
\setkeys{Gin}{width=\textwidth} % По умолчанию размер всех добавляемых рисунков будет подгоняться под ширину текста
\renewcommand{\thesubfigure}{\asbuk{subfigure}} % Нумерация подрисунков строчными буквами кириллицы
%\setlength{\abovecaptionskip}{0pt} % Отбивка над подписью - тут не меняем
%\setlength{\belowcaptionskip}{0pt} % Отбивка под подписью - тут не меняем
\usepackage[section]{placeins} % Объекты типа float (рисунки/таблицы) не вылезают за границы секциии, в которой они объявлены

%%% Задаем параметры ссылок и гиперссылок %%% 

\usepackage{hyperref}                               % Подгружаем нужный пакет
\hypersetup{
    colorlinks=true,                                % Все ссылки и гиперссылки цветные
    linktoc=all,                                    % В оглавлении ссылки подключатся для всех отображаемых уровней
    linktocpage=true,                               % Ссылка - только номер страницы, а не весь заголовок (так выглядит аккуратнее)
    linkcolor=black,                                  % Цвет ссылок и гиперссылок - красный
    citecolor=black                                   % Цвет цитировний - красный
}

%%% Настраиваем отображение списков %%%

\usepackage{enumitem}                               % Подгружаем пакет для гибкой настройки списков
\renewcommand*{\labelitemi}{\normalfont{--}}        % В ненумерованных списках для пунктов используем короткое тире
\makeatletter
    \AddEnumerateCounter{\asbuk}{\russian@alph}     % Объясняем пакету enumitem, как использовать asbuk
\makeatother
\renewcommand{\labelenumii}{\asbuk{enumii})}        % Кириллица для второго уровня нумерации
\renewcommand{\labelenumiii}{\arabic{enumiii})}     % Арабские цифры для третьего уровня нумерации
\setlist{noitemsep, leftmargin=*}                   % Убираем интервалы между пунками одного уровня в списке
\setlist[1]{labelindent=\parindent}                 % Отступ у пунктов списка равен абзацному отступу
\setlist[2]{leftmargin=\parindent}                  % Плюс еще один такой же отступ для следующего уровня
\setlist[3]{leftmargin=\parindent}                  % И еще один для третьего уровня

%%% Счетчики для нумерации объектов %%%

\counterwithout{figure}{chapter}                    % Сквозная нумерация рисунков по документу
\counterwithout{equation}{chapter}                  % Сквозная нумерация математических выражений по документу
\counterwithout{table}{chapter}                     % Сквозная нумерация таблиц по документу

%%% Реализация библиографии пакетами biblatex и biblatex-gost с использованием движка biber %%%

\usepackage{csquotes} % Пакет для оформления сложных блоков цитирования (biblatex рекомендует его подключать)
\usepackage[%
backend=biber,                                      % Движок
bibencoding=utf8,                                   % Кодировка bib-файла
sorting=none,                                       % Настройка сортировки списка литературы
style=gost-numeric,                                 % Стиль цитирования и библиографии по ГОСТ
language=auto,                                      % Язык для каждой библиографической записи задается отдельно
autolang=other,                                     % Поддержка многоязычной библиографии
sortcites=true,                                     % Если в квадратных скобках несколько ссылок, то отображаться будут отсортированно
movenames=false,                                    % Не перемещать имена, они всегда в начале библиографической записи
maxnames=5,                                         % Максимальное отображаемое число авторов
minnames=3,                                         % До скольки сокращать число авторов, если их больше максимума
doi=false,                                          % Не отображать ссылки на DOI
isbn=false,                                         % Не показывать ISBN, ISSN, ISRN
]{biblatex}[2016/09/17]
\DeclareDelimFormat{bibinitdelim}{}                 % Убираем пробел между инициалами (Иванов И.И. вместо Иванов И. И.)
\addbibresource{biba.bib}                           % Определяем файл с библиографией

%%% Скрипт, который автоматически подбирает язык (и, следовательно, формат) для каждой библиографической записи %%%
%%% Если в названии работы есть кириллица - меняем значение поля langid на russian %%%
%%% Все оставшиеся пустые места в поле langid заменяем на english %%%

% \DeclareSourcemap{
%   \maps[datatype=bibtex]{
%     \map{
%         \step[fieldsource=title, match=\regexp{^\P{Cyrillic}*\p{Cyrillic}.*}, final]
%         \step[fieldset=langid, fieldvalue={russian}]
%     }
%     \map{
%         \step[fieldset=langid, fieldvalue={english}]
%     }
%   }
% }

%%% Прочие пакеты для расширения функционала %%%

\usepackage{longtable,ltcaption}                    % Длинные таблицы
\usepackage{multirow,makecell}                      % Улучшенное форматирование таблиц
\usepackage{booktabs}                               % Еще один пакет для красивых таблиц
\usepackage{soul}                                   % Поддержка переносоустойчивых подчёркиваний и зачёркиваний
\usepackage{icomma}                                 % Запятая в десятичных дробях
\usepackage{hyphenat}                               % Для красивых переносов
\usepackage{textcomp}                               % Поддержка "сложных" печатных символов типа значков иены, копирайта и т.д.
\usepackage[version=4]{mhchem}                      % Красивые химические уравнения
\usepackage{amsmath}                                % Усовершенствование отображения математических выражений 

%%% Вставляем по очереди все содержательные части документа %%%

\begin{document}

\thispagestyle{empty}

\begin{center}
    МИНИСТЕРСТВО ОБРАЗОВАНИЯ РЕСПУБЛИКИ БЕЛАРУСЬ \\ 
    БЕЛОРУССКИЙ ГОСУДАРСТВЕННЫЙ УНИВЕРСИТЕТ \\ 
    ФИЗИЧЕСКИЙ ФАКУЛЬТЕТ \\
    Кафедра компьютерного моделирования
\end{center}

\vspace{100pt}

\begin{center}
    \textbf{ФИЗИЧЕСКИ-ИНФОРМИРОВАННЫЕ НЕЙРОННЫЕ СЕТИ ДЛЯ РЕШЕНИЯ ЗАДАЧ ГИДРОДИНАМИКИ}
    
    \hspace{10mm}
    \textbf{Курсовая работа}
    \hspace{10mm}
    
    Специальность 1-31 04 08 Компьютерная физика
\end{center}

\vfill

\begin{flushright}
    \textbf{Исполнитель:} \\
    студент III курса 4 группы \\
    Степанов Игорь Дмитриевич \\
    \vspace{10mm}    
    \textbf{Научный руководитель:} \\
    старший преподаватель \\
    Тимощенко Игорь Андреевич
\end{flushright}

\vfill

\begin{center}
    Минск, 2024
\end{center}                                     % Титульник

\newpage % Переходим на новую страницу
\setcounter{page}{2} % Начинаем считать номера страниц со второй
\OnehalfSpacing* % Задаем полуторный интервал текста (в титульнике одинарный, поэтому команда стоит после него)

\tableofcontents*                                   % Автособираемое оглавление

\newpage

\chapter{Обзор библиотеки DeepXDE \cite{lu2021deepxde}}

DeepXDE (Deep Learning for Differential Equations) –-- это открытая библиотека для Python,
предназначенная для решения различных типов дифференциальных уравнений с помощью методов
глубокого обучения. Она была разработана группой исследователей из Научно-технического
университета Китая и представлена в 2019 году. DeepXDE позволяет решать задачи, описываемые
обыкновенными и частными дифференциальными уравнениями, включая уравнения в частных производных,
интегро-дифференциальные уравнения и уравнения с переменными коэффициентами.

Данная библиотека поддерживает следущие крупные библиотеки машинного обучения: tensorflow.compat.v1,
tensorflow, pytorch, jax, paddle.

\section{Возможности библиотеки}
Для постановки физической задачи необходимо четко определить, что рассматривается в качестве системы
(тело, частица, сплошная среда и т.д.), а также ее границы и взаимодействие с окружающей средой.
Следует задать начальное состояние системы, такое как начальное положение, скорость, температура,
давление и т.д. Необходимо определить, какие физические законы и принципы применимы к данной системе
(законы Ньютона, законы сохранения, принципы термодинамики и т.д.), и записать уравнения, описывающие
движение, взаимодействие или другие процессы в системе, на основе выбранных законов и принципов. При
необходимости нужно задать дополнительные условия, такие как связи, граничные условия, свойства
материалов и т.д. Также следует четко сформулировать, какие физические величины необходимо определить
в результате решения задачи. Этот минимум информации позволяет корректно сформулировать физическую
задачу и создать математическую модель для ее решения.

Суть метода заключается в использовании невязки всех уравнений, граничных и начальных условий в качестве
функции потерь. Это позволяет нейронной сети минимизировать данную ошибку и получить необходимый результат.

Невязка - это разница между левым и правым значениями уравнений. Она рассчитывается на основе начального решения,
которое задается случайным образом. Затем, с помощью методов минимизации функции потерь, решение уточняется.

Аналогичный метод решения мы можем видеть при численном решении системы линейных алгебраических уравнений (СЛАУ)
с помощью градиентного спуска. В этом случае, невязка также используется в качестве функции потерь, но минимизирует
ее Градиентный спуск, чтобы получить решение СЛАУ.

Градиентный спуск - это один из наиболее распространенных методов минимизации функции потерь. Он заключается в том,
что на каждом шаге алгоритма мы изменяем решение в направлении, которое уменьшает значение функции потерь.

\subsection{Область}
Библиотека DeepXDE имеет ряд стандартных областей (geometry), которые можно применить к большинству задач:
\begin{minted}{python}
from deepxde.geometry.geometry_1d import Interval
from deepxde.geometry.geometry_2d import (
    Disk, Ellipse, Polygon, Rectangle, StarShaped, Triangle
)
from deepxde.geometry.geometry_3d import Cuboid, Sphere
from deepxde.geometry.geometry_nd import Hypercube, Hypersphere
from deepxde.geometry.timedomain  import GeometryXTime, TimeDomain
\end{minted}
Объекты данных классов имеют следущий функционал (весь функционал унаследован от интерфейсного класса deepxde.geometry.Geometry):
\begin{itemize}
    \item Проверить, принадлежит ли точка данной области или ее границе
    \item Нормаль к границе.
    \item Объединение, пересечение и интерсекцию с другими обдостями.
    \item Получить набор случайных точек в данной области или на ее границе.
\end{itemize}
Эти возможности предоставляют полный доступ к созданию и использованию области для решения
дифференциальных уравнений.

Например для рассмотрения обтекания бесконечного цилиндра нам потребуется следущая область:
\begin{minted}{python}
base_domain = Rectangle(xmin=[0, 0], xmax=[3, 1])
barrier_domain = Ellipse(
    center=[0.5, 0.5], semimajor=0.1, semiminor=0.1
    )
space_domain = base_domain - barrier_domain

time_domain = dde.geometry.TimeDomain(0, 1)

domain = dde.geometry.GeometryXTime(space_domain, time_domain)
\end{minted}
Далее мы будем использовать данную область для примера решенния задачи.
\subsection{Уравнения}
Для задания уравнения или системы уравнений мы можем использовать следущие два оператора:
\begin{minted}{python}
deepxde.grad.jacobian
deepxde.grad.hessian
\end{minted}
Они представляют собой операторы набла и Лапласа соответственно. Для использования уравнений
в дальнейшем необходимо создать функцию, которая будет возвращать уравнение или массив уравнений.

Для задач гидродинамики воспользуемся уравнения Навье-Стокса. Пример реализации уравнений
для двумерной задачи:
\begin{minted}{python}
def navier_stocks(x, u):
    u_vel, v_vel, p = u[:, 0:1], u[:, 1:2], u[:, 2:3]

    u_vel_x = dde.grad.jacobian(u, x, i=0, j=0)
    u_vel_y = dde.grad.jacobian(u, x, i=0, j=1)
    u_vel_t = dde.grad.jacobian(u, x, i=0, j=2)
    u_vel_xx = dde.grad.hessian(u, x, component=0, i=0, j=0)
    u_vel_yy = dde.grad.hessian(u, x, component=0, i=1, j=1)

    v_vel_x = dde.grad.jacobian(u, x, i=1, j=0)
    v_vel_y = dde.grad.jacobian(u, x, i=1, j=1)
    v_vel_t = dde.grad.jacobian(u, x, i=1, j=2)
    v_vel_xx = dde.grad.hessian(u, x, component=1, i=0, j=0)
    v_vel_yy = dde.grad.hessian(u, x, component=1, i=1, j=1)

    p_x = dde.grad.jacobian(u, x, i=2, j=0)
    p_y = dde.grad.jacobian(u, x, i=2, j=1)

    momentum_x = (
        u_vel_t
        + (u_vel * u_vel_x + v_vel * u_vel_y)
        + p_x
        - 1 / Re * (u_vel_xx + u_vel_yy)
    )
    momentum_y = (
        v_vel_t
        + (u_vel * v_vel_x + v_vel * v_vel_y)
        + p_y
        - 1 / Re * (v_vel_xx + v_vel_yy)
    )
    continuity = u_vel_x + v_vel_y
    return [momentum_x, momentum_y, continuity]
\end{minted}

\subsection{Граничные и начальные условия}
Для задания граничных условий используются следущие классы в DeepXDE (подпространство deepxde\.icbc):
\begin{itemize}
\item DirichletBC
\item NeumannBC
\item OperatorBC
\item RobinBC
\item PeriodicBC
\item PointSetBC
\item PointSetOperatorBC
\end{itemize}

Каждый из них задает соответствующее условие, а также позволяет унаследовать и изменить условия
для определения границы с помощью вспомогательного аргумента.

Для начальных условий из того же подпространства есть класс IC, который задает значение нашего решения в начальный
момент времени и также, как и граничные условия, может переопределять условия для границы.

Пример реализации условий для уравнения Навье-Стокса в двумерной задаче:
\begin{minted}{python}
boundary_condition_u = dde.DirichletBC(
    domain, lambda x: 1, 
    lambda x, on_boundary: base_domain.on_boundary(x[0:2]),
    component=0
)
boundary_condition_v = dde.DirichletBC(
    domain, lambda x: 0, 
    lambda x, on_boundary: base_domain.on_boundary(x[0:2]),
    component=1
)
barrier_condition_u = dde.DirichletBC(
    domain, lambda x: 0, 
    lambda x, on_boundary: barrier_domain.on_boundary(x[0:2]),
    component=0
)
barrier_condition_v = dde.DirichletBC(
    domain, lambda x: 0, 
    lambda x, on_boundary: barrier_domain.on_boundary(x[0:2]),
    component=1
)

initial_condition_u = dde.IC(
    domain, lambda x: 0,
    lambda x, on_initial: on_initial,
    component=0
)
initial_condition_v = dde.IC(
    domain, lambda x: 0,
    lambda x, on_initial: on_initial, 
    component=1
)
\end{minted}
    
\subsection{Создание модели}
Пример:
\begin{minted}{python}
data = dde.data.TimePDE(
    domain,
    navier_stocks,
    [
        boundary_condition_u,
        boundary_condition_v,
        initial_condition_u,
        initial_condition_v,
        barrier_condition_u,
        barrier_condition_v,
    ],
    num_domain=50000,
    num_boundary=5000,
    num_initial=5000,
    num_test=10000,
)

net = dde.nn.FNN([3] + 4 * [50] + [3], "tanh", "Glorot normal")

model = dde.Model(data, net)    
\end{minted}
\subsection{Обучение модели}
Пример:
\begin{minted}{python}
model.compile("adam", lr=1e-3, loss_weights=[1, 1, 100, 10, 10, 10, 10, 10, 10])
model.train(iterations=3000, display_every=1)
model.compile("L-BFGS", loss_weights=[1, 1, 100, 10, 10, 10, 10, 10, 10])
losshistory, train_state = model.train()
\end{minted}
Также имеется возможность сохранять и загружать существующую модель:
\begin{minted}{python}
model.train(iterations=3000, display_every=1, model_save_path="model/")

model.restore("model/good_model.ckpt-43904.ckpt", verbose=1)    
\end{minted}
\subsection{Применение модели}
Для создания модели требуется четкое понимание функции ошибки и оптимизаторов, способных минимизировать данную функцию.
Так как мы не можем гарантированно знать нашу функцию ошибки, остается только эксперементировать с различными
оптимизаторами.

В данной библиотеке доступны все оптимизаторы из соответствующей оболочки для нейронных сетей. В данной работе преимущественно
используется Tensorflow, поэтому доступны следущие оптимизаторы

\begin{itemize}
    \item Adadelta
    \item Adafactor
    \item Adagrad
    \item Adam
    \item AdamW
    \item Adamax
    \item Ftrl
    \item Lion
    \item Nadam
    \item RMSprop
    \item SGD
\end{itemize}
А также L-BFGS и L-BFGS-B предоставляемые самой библиотекой DeepXDE.
Для остальных доступных оболочек соответствующие им оптимизаторы можно найти тут tensorflow.compat.v1 \cite{tfv1opt},
tensorflow \cite{tfopt}, pytorch \cite{pytorchopt}, jax \cite{jaxopt}, paddle \cite{paddleopt}.

Далее нужно оценить весовые коэффициенты для каждого из входных условий. Например, в последущих примерах я отдаю предпочтение
выполнению граничных условий, поэтому они идут с весом 100, и уравнение непрерывности с весом 10, остальное 1.



\chapter{Использование NPU для ускорения работы модели}

С точки зрения возможности использования NPU следует выбирать tensorflow или tensorflow.compat.v1,
так как они поддерживают экспорт модели в TensorFlow Lite и их использование на устройствах Android
и iOS\cite{nnapi}. Такой подход можно использовать для реалистичного моделирования физики как в научных целях, так и
в мобильных играх.

Для реализации приложения с такими возм

\printbibliography[title=Список использованных источников] % Автособираемый список литературы


\end{document}