\chapter{Обзор библиотеки DeepXDE \cite{lu2021deepxde}}

DeepXDE (Deep Learning for Differential Equations) –-- это открытая библиотека для Python,
предназначенная для решения различных типов дифференциальных уравнений с помощью методов
глубокого обучения. Она была разработана группой исследователей из Научно-технического
университета Китая и представлена в 2019 году. DeepXDE позволяет решать задачи, описываемые
обыкновенными и частными дифференциальными уравнениями, включая уравнения в частных производных,
интегро-дифференциальные уравнения и уравнения с переменными коэффициентами.

Данная библиотека поддерживает следущие крупные библиотеки машинного обучения: tensorflow.compat.v1,
tensorflow, pytorch, jax, paddle.

\section{Возможности библиотеки}
Для постановки физической задачи необходимо четко определить, что рассматривается в качестве системы
(тело, частица, сплошная среда и т.д.), а также ее границы и взаимодействие с окружающей средой.
Следует задать начальное состояние системы, такое как начальное положение, скорость, температура,
давление и т.д. Необходимо определить, какие физические законы и принципы применимы к данной системе
(законы Ньютона, законы сохранения, принципы термодинамики и т.д.), и записать уравнения, описывающие
движение, взаимодействие или другие процессы в системе, на основе выбранных законов и принципов. При
необходимости нужно задать дополнительные условия, такие как связи, граничные условия, свойства
материалов и т.д. Также следует четко сформулировать, какие физические величины необходимо определить
в результате решения задачи. Этот минимум информации позволяет корректно сформулировать физическую
задачу и создать математическую модель для ее решения.

\subsection{Область}
Библиотека DeepXDE имеет ряд стандартных областей (geometry), которые можно применить к большинству задач.
В общей сложности имеется 3 базовых класса:
\begin{minted}{python}
deepxde.geometry.geometry_1d
deepxde.geometry.geometry_2d
deepxde.geometry.geometry_3d
\end{minted}
от которых унаследованны такие области, как
\begin{minted}{python}
deepxde.geometry.rectangle
deepxde.geometry.circle
deepxde.geometry.ellipse
подправить
\end{minted}
Помимо пространственных координат, можно вводить временную координату, для этого используем
\begin{minted}{python}
    deepxde.geometry.TimeDomain
    \end{minted}
\subsection{Уравнения}

\subsection{Граничные условия}

\subsection{Начальные условия}

\subsection{Создание модели}

\subsection{Обучение модели}

\subsection{Применение модели}

\subsection{Визуализация решения}


\chapter{Использование NPU телефона для ускорения работы модели \cite{nnapi}}

С точки зрения возможности использования NPU следует выбирать tensorflow или tensorflow.compat.v1,
так как они поддерживают экспорт модели в TensorFlow Lite и их использование на устройствах Android
и iOS. Такой подход можно использовать для реалистичного моделирования физики как в научных целях, так и
в мобильных играх.

Для реализации приложения с такими возм

% \chapter{First}

% Дана цилиндрическая емкость с гибким дном. 
% Дно колебается по закону 
% $z = A cos(\frac{\pi r}{2R}) sin(2\pi\nu t)$,
% где $A$ -- амплитуда колебаний, $r$ -- удаление 
% точки от центра сечения цилиндра, $R$ -- радиус 
% цилиндра, $\mu$ -- частота колебаний, $t$ -- время.
% Промоделировать волны на поверхности воды.

% Решение: \\
% Нормаль к нижней части запишется как:
% $$
%     \vec{n} = 
%     \left\{ 
%             2A \text{sin}\left( \frac{\pi r_0}{2R} \right)\text{sin}\left( 2\pi \nu t \right), \\
%             0, \\
%             \frac{\pi}{R}
%     \right\}        
% $$
% Берем уравнение Навье-Стокса 
% \begin{equation}
%     \rho \left[\frac{\partial{\vec{v}}}{\partial{t}}+\left(\vec{v}\cdot\vec{\nabla}\right)\vec{v}\right] = 
%     -\text{grad}p + \eta \Delta\vec{v} + (\zeta + \frac{\eta}{3})\text{grad}\text{div}\vec{v}
% \end{equation}
% Будем считать жидкость несжимаемой, тогда:
% \begin{equation}
%     \text{div}\vec{v} = 0
% \end{equation}
% \begin{equation}
%     \frac{\partial{\vec{v}}}{\partial{t}}+\left(\vec{v}\cdot\vec{\nabla}\right)\vec{v} = 
%     \frac{-\text{grad}p}{\rho} + \frac{\eta}{\rho} \Delta\vec{v}
% \end{equation}
% Перепишем данное уравнение в цилиндрической системе:
% \begin{equation}
%     \begin{cases}
%         x = r cos(\varphi) \\
%         y = r sin(\varphi) \\
%         z = z 
%     \end{cases}
% \end{equation}
% \begin{equation}
%     \begin{cases}
%         \frac{\partial{v_r}}{\partial t} + \left(\vec{v}\cdot\vec{\nabla}\right)v_r - \frac{v_\varphi^2}{r} = -\frac{1}{R} \frac{\partial p}{\partial r} + \nu \left( \Delta v_r - \frac{v_r}{r^2} - \frac{2}{r^2} \frac{\partial v_\varphi}{\partial \varphi} \right) \\
%         \frac{\partial{v_\varphi}}{\partial t} + \left(\vec{v}\cdot\vec{\nabla}\right)v_\varphi - \frac{v_\varphi v_r}{r} = -\frac{1}{Rr} \frac{\partial p}{\partial \varphi} + \nu \left( \Delta v_\varphi - \frac{v_\varphi}{r^2} + \frac{2}{r^2} \frac{\partial v_r}{\partial \varphi} \right) \\
%         \frac{\partial{v_z}}{\partial t} + \left(\vec{v}\cdot\vec{\nabla}\right)v_z = -\frac{1}{R} \frac{\partial p}{\partial z} + \nu\Delta v_z \\
%         \left( \vec{v} \cdot\vec{\nabla} \right)f = v_r\frac{\partial f}{\partial r} + \frac{v_\varphi}{r}\frac{\partial f}{\partial \varphi} + v_z \frac{\partial f}{\partial z} \\
%         \Delta f = \frac{1}{r}\frac{\partial}{\partial r}\left( r\frac{\partial f}{\partial r} \right) + \frac{1}{r^2}\frac{\partial^2 f}{\partial \varphi^2} + \frac{\partial^2 f}{\partial z^2}
%     \end{cases}
% \end{equation}
% В данной задаче есть центральная симметрия, движение потенциально, а значит:
% \begin{equation}
%     \begin{cases}
%         v_\varphi = 0 \\
%         \vec{v} = \vec{v}\left( r, z \right)
%     \end{cases}
% \end{equation}
% Итого получаем:
% \begin{equation}
%     \begin{cases}
%         \frac{\partial{v_r}}{\partial t} + v_r \frac{\partial v_r}{\partial r} + v_z \frac{\partial v_r}{\partial z} = -\frac{1}{R} \frac{\partial p}{\partial r} + \nu \left( \frac{1}{r}\frac{\partial}{\partial r}\left( r\frac{\partial v_r}{\partial r} \right) + \frac{\partial^2 v_r}{\partial z^2} - \frac{v_r}{r^2} \right) \\
%         \frac{\partial{v_z}}{\partial t} + v_r \frac{\partial v_z}{\partial r} + v_z \frac{\partial v_z}{\partial z} = -\frac{1}{R} \frac{\partial p}{\partial z} + \nu\left( \frac{1}{r}\frac{\partial}{\partial r}\left( r\frac{\partial v_z}{\partial r} \right) + \frac{\partial^2 v_z}{\partial z^2} \right) \\
%         \frac{1}{r}\frac{\partial \left( r v_r \right)}{\partial r} + \frac{\partial v_z}{\partial z} = 0
%     \end{cases}
% \end{equation}
% И граничные условия:
% \begin{equation}
%     \begin{cases}
%         \vec{v}\left( R, z \right) = \vec{0} \\
%         \vec{v}\left( r, 0 \right) = \left\{\frac{A^2 \pi^2 \nu}{4R}sin\left( \frac{\pi r}{R} \right)sin\left( 4\pi \nu t\right), 0, 2A\pi\nu cos\left( \frac{\pi r}{2R} \right)cos\left( 2\pi \nu t\right) \right\}
%     \end{cases}
% \end{equation}
% На поверхности:
% \begin{equation}
%     \begin{cases}
%         \sigma_{ij} = \eta \left( \frac{\partial v_i}{\partial x_j} + \frac{\partial v_j}{\partial x_i} \right) \\
%         p + \sigma_{zz} = p_{\text{газа}} \\
%         \vec{v}_{\text{газа}} \cdot \vec{n} = \vec{v}_{\text{жидкости}} \cdot \vec{n}
%     \end{cases}
% \end{equation}
% Пусть раздел двух сред будет на высоте h, высота всей емкости H, тогда задача сводится к двум системам: \\
% Для $ 0 \le z < h $
% \begin{equation}
%     \begin{cases}
        
%     \end{cases}
% \end{equation}
% Для $ h < z \le H $
