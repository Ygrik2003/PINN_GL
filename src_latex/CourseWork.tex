\chapter{First}

Дана цилиндрическая емкость с гибким дном. 
Дно колебается по закону 
$z = A cos(\frac{\pi r}{2R}) sin(2\pi\nu t)$,
где $A$ -- амплитуда колебаний, $r$ -- удаление 
точки от центра сечения цилиндра, $R$ -- радиус 
цилиндра, $\mu$ -- частота колебаний, $t$ -- время.
Промоделировать волны на поверхности воды.

Решение: \\
Нормаль к нижней части запишется как:
$$
    \vec{n} = 
    \left\{ 
            2A \text{sin}\left( \frac{\pi r_0}{2R} \right)\text{sin}\left( 2\pi \nu t \right), \\
            0, \\
            \frac{\pi}{R}
    \right\}        
$$
Берем уравнение Навье-Стокса 
\begin{equation}
    \rho \left[\frac{\partial{\vec{v}}}{\partial{t}}+\left(\vec{v}\cdot\vec{\nabla}\right)\vec{v}\right] = 
    -\text{grad}p + \eta \Delta\vec{v} + (\zeta + \frac{\eta}{3})\text{grad}\text{div}\vec{v}
\end{equation}
Будем считать жидкость несжимаемой, тогда:
\begin{equation}
    \text{div}\vec{v} = 0
\end{equation}
\begin{equation}
    \frac{\partial{\vec{v}}}{\partial{t}}+\left(\vec{v}\cdot\vec{\nabla}\right)\vec{v} = 
    \frac{-\text{grad}p}{\rho} + \frac{\eta}{\rho} \Delta\vec{v}
\end{equation}
Перепишем данное уравнение в цилиндрической системе:
\begin{equation}
    \begin{cases}
        x = r cos(\varphi) \\
        y = r sin(\varphi) \\
        z = z 
    \end{cases}
\end{equation}
\begin{equation}
    \begin{cases}
        \frac{\partial{v_r}}{\partial t} + \left(\vec{v}\cdot\vec{\nabla}\right)v_r - \frac{v_\varphi^2}{r} = -\frac{1}{R} \frac{\partial p}{\partial r} + \nu \left( \Delta v_r - \frac{v_r}{r^2} - \frac{2}{r^2} \frac{\partial v_\varphi}{\partial \varphi} \right) \\
        \frac{\partial{v_\varphi}}{\partial t} + \left(\vec{v}\cdot\vec{\nabla}\right)v_\varphi - \frac{v_\varphi v_r}{r} = -\frac{1}{Rr} \frac{\partial p}{\partial \varphi} + \nu \left( \Delta v_\varphi - \frac{v_\varphi}{r^2} + \frac{2}{r^2} \frac{\partial v_r}{\partial \varphi} \right) \\
        \frac{\partial{v_z}}{\partial t} + \left(\vec{v}\cdot\vec{\nabla}\right)v_z = -\frac{1}{R} \frac{\partial p}{\partial z} + \nu\Delta v_z \\
        \left( \vec{v} \cdot\vec{\nabla} \right)f = v_r\frac{\partial f}{\partial r} + \frac{v_\varphi}{r}\frac{\partial f}{\partial \varphi} + v_z \frac{\partial f}{\partial z} \\
        \Delta f = \frac{1}{r}\frac{\partial}{\partial r}\left( r\frac{\partial f}{\partial r} \right) + \frac{1}{r^2}\frac{\partial^2 f}{\partial \varphi^2} + \frac{\partial^2 f}{\partial z^2}
    \end{cases}
\end{equation}
В данной задаче есть центральная симметрия, движение потенциально, а значит:
\begin{equation}
    \begin{cases}
        v_\varphi = 0 \\
        \vec{v} = \vec{v}\left( r, z \right)
    \end{cases}
\end{equation}
Итого получаем:
\begin{equation}
    \begin{cases}
        \frac{\partial{v_r}}{\partial t} + v_r \frac{\partial v_r}{\partial r} + v_z \frac{\partial v_r}{\partial z} = -\frac{1}{R} \frac{\partial p}{\partial r} + \nu \left( \frac{1}{r}\frac{\partial}{\partial r}\left( r\frac{\partial v_r}{\partial r} \right) + \frac{\partial^2 v_r}{\partial z^2} - \frac{v_r}{r^2} \right) \\
        \frac{\partial{v_z}}{\partial t} + v_r \frac{\partial v_z}{\partial r} + v_z \frac{\partial v_z}{\partial z} = -\frac{1}{R} \frac{\partial p}{\partial z} + \nu\left( \frac{1}{r}\frac{\partial}{\partial r}\left( r\frac{\partial v_z}{\partial r} \right) + \frac{\partial^2 v_z}{\partial z^2} \right) \\
        \frac{1}{r}\frac{\partial \left( r v_r \right)}{\partial r} + \frac{\partial v_z}{\partial z} = 0
    \end{cases}
\end{equation}
И граничные условия:
\begin{equation}
    \begin{cases}
        \vec{v}\left( R, z \right) = \vec{0} \\
        \vec{v}\left( r, 0 \right) = \left\{\frac{A^2 \pi^2 \nu}{4R}sin\left( \frac{\pi r}{R} \right)sin\left( 4\pi \nu t\right), 0, 2A\pi\nu cos\left( \frac{\pi r}{2R} \right)cos\left( 2\pi \nu t\right) \right\}
    \end{cases}
\end{equation}
На поверхности:
\begin{equation}
    \begin{cases}
        \sigma_{ij} = \eta \left( \frac{\partial v_i}{\partial x_j} + \frac{\partial v_j}{\partial x_i} \right) \\
        p + \sigma_{zz} = p_{\text{газа}} \\
        \vec{v}_{\text{газа}} \cdot \vec{n} = \vec{v}_{\text{жидкости}} \cdot \vec{n}
    \end{cases}
\end{equation}
Пусть раздел двух сред будет на высоте h, высота всей емкости H, тогда задача сводится к двум системам: \\
Для $ 0 \le z < h $
\begin{equation}
    \begin{cases}
        
    \end{cases}
\end{equation}
Для $ h < z \le H $
